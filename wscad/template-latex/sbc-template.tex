\documentclass[12pt]{article}

\usepackage{sbc-template}

\usepackage{graphicx,url}

%\usepackage[brazil]{babel}   
\usepackage[latin1]{inputenc}  

\usepackage{amsmath}
\usepackage{amssymb} 
\usepackage{mathtools}

\usepackage{algorithm}
\usepackage[noend]{algpseudocode}

 
\newcommand{\Cfield}{\mathbb{C}}
\newcommand{\Rfield}{\mathbb{R}}

\newcommand{\norm}[1]{\left\lVert#1\right\rVert}

\sloppy

\title{Optimizing a Boundary Elements Method implementations with GPU}

\author{Giuliano A. F. Belinassi\inst{1}, Rodrigo Siqueira\inst{1}, Ronaldo Carrion\inst{2}
  Alfredo Goldman\inst{1} }


\address{Instituto de Matem�tica e Estat�stica (IME) -- Universidade de S�o Paulo
  (USP)\\
  Rua do Mat�o, 1010 -- S�o Paulo -- SP -- Brazil
\nextinstitute
  Escola Polit�cnica (EP)  -- Universidade de S�o Paulo\\
  Avenue Professor Mello Moraes, 2603 -- S�o Paulo -- SP -- Brazil
}

\begin{document} 

\maketitle

\begin{abstract}
  This meta-paper describes the style to be used in articles and short papers
  for SBC conferences. For papers in English, you should add just an abstract
  while for the papers in Portuguese, we also ask for an abstract in
  Portuguese (``resumo''). In both cases, abstracts should not have more than
  10 lines and must be in the first page of the paper.
\end{abstract}
     
\section{Introduction}

Since the computer was proven to be a useful machine, there always has been an interest of 
building faster versions of it to solve bigger and more complex problems in a matter 
of time that no human can. One way archiving this is by building more complex sequential 
CPUs, with more transistors. Recently, the size of CPU components got so small that it is 
becoming difficult to create such faster sequential CPUs \cite{brock:2006}. As a consequence of this fact, 
CPU manufacturers are investing in multicore CPUs, capable of running one than one task 
asynchronously.

Parallel computing is hard to define, but intuitively, it is a computation method 
that allows data to be distributed and processed simultaneously. In Flynn's taxonomy
\cite{pacheco:2011},
there are two types of parallel computer archictectures: 

\begin{enumerate}
\item Single Instruction, Multiple Data (SIMD); a processor that allows a chunk of
data to be loaded and a single instruction to be used to process it. As example of SIMD,
there is Intel's SSE \cite{sse} and Graphics Processing Units (GPU) are examples it. 

\item Multiple Instruction, Multiple Data (MIMD); a system consisting of
multiple independent processing units, executing asynchronously. Multicore CPUs are a 
example of MIMD.
\end{enumerate}

Originally, GPU was designed to render graphics in real time and OpenGL and DirectX 
were the first libraries designed to access GPUs resources and provide graphics. 
However, researchers realized that GPUs could be used for other applications different 
from graphics. Consequently, graphical libraries were used by engineers and scientists 
in their specific problems, however, they had to convert their problem into a graphical 
domain.

NVIDIA noticed a new demand for their products and created an API called CUDA to enable 
the use of GPUs in general purpose situation. CUDA has the concept of kernels, which are
 functions called from host to be executed in 
the GPU threads. Kernels are organized into a set of blocks wherein each block is a set 
of threads that cooperate with each other \cite{patterson:2007}.

GPU's memory is divided into global memory, local memory, and shared memory. First, it is 
a memory that all threads can access. Second, it is a memory that is private to a thread. 
Third, it is a  low-latency memory that is shared between all threads in the same block.
\cite{patterson:2007}. 


%TODO:
The Boundary Elements Method (BEM) is a computational method of solving linear partial 
differentials equations that have been formulated as integral equations.
This is used in many engineering areas, one of them is to analyze waves propagation on the 
ground and its effects on nearby structures. It requires high computational power and this 
article discusses how GPUs can be used to accelerate its calculations.

Given a sequential implementation of BEM by \cite{carrion:02}, the main objectives
of this project includes the creation of automated tests to check if the modified 
program results are numerically compatible with the original version; somewhat 
modernize the legacy code, which was written in Fortran 77; optimize the code, 
removing repeated unnecessary calculations and managing a better usage of the 
Central Processing Unit (CPU) resources; 
identify the most time-consuming subroutines and paralellize them.


%Let now $\hat{x} \in \Cfield^n$
%be a value that should have been $x$, but contains errors due to whatever
%reasons. One can measure \textit{how far} $\hat{x}$ is from $x$ by computing
%$\norm{x - \hat{x}}$. If the vector $\infty$-norm is used, then the result yielded
%is the maximum difference between a single entry of the vector. If the matrix
%$\infty$-norm is used, then the result is the biggest sum of errors of a matrix
%row. The matrix 1-norm yields the biggest sum of erros of a matrix column.

\section{Parallelization Technique}

A parallel implementation of BEM began by analyzing and modifying a sequential code 
provided by Carrion. Gprof \cite{binutils}, a profiling tool by GNU, showed that the most time-consuming 
routine was Nonsingd, a subroutine designed to solve small parts of the nonsingular dynamic 
problem. Since most calls to Nonsingd were from Ghmatecd, most of the parallelization effort 
was focused on that last routine.

\subsection{Ghmatecd Parallelization}
Ghmatecd works in the following way: It assembles two complex matrices H and G by computing 
smaller $3 \times 3$ matrices returned by Nonsingd and Sigmaec. Let $n$ be the number of 
mesh elements and $m$ be the number of bondary elements. Algorithm 1 illustrates how Ghmatecd 
works.

\begin{algorithm}
\caption{Creates $H, G \in \Cfield^{(3m)\times(3n)}$}\label{euclid}
\begin{algorithmic}[1]
	\Procedure{Ghmatecd}{}
		\For{$j := 1, n$} 
			\For{$i := 1, m$}
				\State{$ii := 3(i-1) + 1;     jj := 3(j-1) + 1$}
				\If{$i == j$}
					\State{$Gelement, Helement \leftarrow \text{Sigmaec}(i)$}\Comment{two $3\times3$ complex matrices}					
				\Else
					\State{$Gelement, Helement \leftarrow \text{Nonsingd}(i, j)$}	
				\EndIf
				\State{$G[ii:ii+2][jj:jj+2] \leftarrow Gelement$}
				\State{$H[ii:ii+2][jj:jj+2] \leftarrow Helement$}
			\EndFor
	 \EndFor
	\EndProcedure
\end{algorithmic}
\end{algorithm}

There is no interdependency between iteractions, so the two nested loop in the algorithm can be computed in parallel, 
hence, both matrices $H$ and $G$ can be built in a parallel fashion. If the number of processors is small, 
then parallelizing the two nested loops in line computing that nested loop in parallel is enough because even for small instances of the problem, 
$n \times m$ will be bigger than the number of processors.
By other hand, GPUs have greater parallel capability than CPUs, and the above strategy alone would generate a waste of 
computational resources. Since Nonsingd is the cause of the high time cost of Ghmatecd, the main effort was to implement
an optimized version of Ghmatecd, called Ghmatecd\_Nonsingd, that only computes the Nonsingd case in the GPU, and leave Sigmaec to be computed in the CPU
after the computation of Ghmatecd\_Nonsingd is completed. 

Let $g$ be the number of Gauss quadrature points. The Algorithm 2 pictures this new strategy.

\begin{algorithm}
\caption{Creates $H, G \in \Cfield^{(3m)\times(3n)}$}\label{euclid}
\begin{algorithmic}[1]
	\Procedure{Ghmatecd\_nonsingd}{}
		\For{$j := 1, n$} 
			\For{$i := 1, m$}
				\State{$ii := 3(i-1) + 1;     jj := 3(j-1) + 1$}
				\State{Allocate \textit{Hbuffer} and \textit{Gbuffer}, buffer of matrices $3 \times 3$ of size $g^2$}
				\If{$i \neq j$}
					\For{$y := 1, g$}
						\For{$x := 1, g$}
							\State{\textit{params} $\leftarrow$ ComputeParameters$(i, j, x, y)$}
							\State{$\textit{Hbuffer}(x, y) \leftarrow \text{GenerateMatrixH}(x, y, params)$}
							\State{$\textit{Gbuffer}(x, y) \leftarrow \text{GenerateMatrixG}(x, y, params)$}
						\EndFor
					\EndFor
				\EndIf
				\State{$Gelement \leftarrow \text{SumAllMatricesInBuffer}(\textit{Gbuffer})$} 
				\State{$Helement \leftarrow \text{SumAllMatricesInBuffer}(\textit{Hbuffer})$}
				\State{$G[ii:ii+2][jj:jj+2] \leftarrow Gelement$}
				\State{$H[ii:ii+2][jj:jj+2] \leftarrow Helement$}
			\EndFor
	 \EndFor
	\EndProcedure

	\Procedure{Ghmatecd\_Sigmaec}{}
		\For{$i := 1, m$}
			\State{$ii := 3(i-1) + 1$}
			\State{$Gelement, Helement \leftarrow \text{Sigmaec}(i)$}	
			\State{$G[ii:ii+2][ii:ii+2] \leftarrow Gelement$}
			\State{$H[ii:ii+2][ii:ii+2] \leftarrow Helement$}
	 \EndFor
	\EndProcedure
	\Procedure{Ghmatecd}{}
		\State{$\text{Ghmatecd\_Nonsingd}()$}
		\State{$\text{Ghmatecd\_Sigmaec}()$}
	\EndProcedure
		
\end{algorithmic}
\end{algorithm}

The Ghmatecd\_Nonsingd can be implemented in CUDA kernel in the following way: Inside a block, create $g \times g$ threads to 
compute in parallel the two nested loop in lines 6 to 7, allocating \textit{Hbuffer} and \textit{Gbuffer} in shared 
memory. Since these buffers contain matrices of size $3 \times 3$, $9$ of these $g \times g$ threads can be used to 
sum all matrices. Notice that $g \geq 3$ is necessary for that condition, and that $g$ is also upper-bounded by the 
amount of shared memory available in the GPU. Launching $m \times n$ blocks to cover the two nested loops in lines
2 to 3 will generate the entire $H$ and $G$ wihout the singular part. The Ghmatecd\_Sigmaec can be parallelized with 
a simple OpenMP Parallel for clause, and it will calculate the remaining $H$ and $G$. 

\subsection{Ghmatece Parallelization}

Ghmatece is a routine designed to create two real matrices H and G associated with the static problem, and it 
is very similar to Ghmatecd. That routine can be implemented in CUDA in the same way as described in Ghmatecd. 

\section{Methodology}

In order to check if the final result obtained by the parallel program is numerically 
compatible with the original, the concept of vector and matrix norms are necessary. 
Let %$x \in \Cfield^n$ and 
$A \in \Cfield^{m \times n}$. \cite{watkins:2004} defines  
matrix 1-norm as:
\begin{equation}
%	\norm{x}_{\infty} = \max\limits_{1 \leq k \leq n} |x_k| \qquad 
%	\norm{A}_{\infty} = \max\limits_{1 \leq i \leq m} \sum_{j=1}^{n} |a_{ij}| \quad
	\norm{A}_{   1  } = \max\limits_{1 \leq j \leq n} \sum_{i=1}^{m} |a_{ij}| \quad
\end{equation}

All norms have the propierty that $\norm{A} = 0$ if and only if $A = 0$.
Let $f$ and $g$ be two numerical 
algorithms that solves the same problem, but in a different fashion. 
Let now $y_f$ be the result computed by $f$ and $y_g$ be the result computed by
$g$. The \textit{error} between those two values can be measured computing
$\norm{y_f - y_g}$.

The error between CPU and GPU versions of $H$ and $G$ matrices were computated by calulating $\norm{H_{cpu} - H_{gpu}}_1$
and $\norm{G_{cpu} - G_{gpu}}_1$. An automated test check if this value is bellow $10^{-4}$.

The elapsed time was computed in seconds with the OpenMP library function OMP\_GET\_WTIME. This function calculates the 
elapsed wall clock time in seconds with double precision.

The GPU time results are a sum of the time elapsed to move the data to GPU, launch and execute the kernel, wait for the 
result, and move the data back to computer's main memory.

For each experiment, there were 4 samples of data, with 240 mesh elements and 100 boundary elements; 960 mesh elements 
and 400 boundary elements; 2160 mesh elements and 900 boundary elements; 4000 mesh elements and 1600 boundary elements. 
All experiments set the Gauss Quadrature Points to 8.

For each experiment execution, it was collected the total time elapsed by the serial code, with OpenMP and CUDA. 
In order to assure the results, all experiments are run 30 times for each sample.
It is important to highlight that before running each experiment, a warmup procedure is executed, that consists of 
running the application with the sample 3 times without collecting any result.

Two computers were used in the experiments. Both machines have a AMD A10-7700K with 4 cores, but one have a 
NVIDIA GeForce GTX980 and another have a GeForce GTX750.

Gfortran 5.4.0 and CUDA 8.0 were used to compile the applications. The main flags used in Gfortran are
-Ofast -funroll-loops -flto . The flags used in
CUDA nvcc compiler are: -use\_fast\_math  -O3  -Xptxas --opt-level=3  -maxrregcount=32 -Xptxas 
--allow-expensive-optimizations=true . 

\section{Results}
The graphic at figure 1 illustrates the results. Since the standard deviation of the results is lower than X, 
plotting the graphic with it is meaningless. 


\section{Future Works}
The current implemented code have limitations. First, there is no logic to assemble H and G by blocks by lauching 
multiple kernels. This strategy would allow bigger problems to be solved, since if the number of mesh elements is 
bigger enough the application would crash due to insufficient video memory.  Second, the singular and nonsingular 
part can be computed independently, so the CPU can be used to compute the singular part while the GPU is computing 
the nonsingular part. The usage of GPUs in the singular case can also be analyzed.

%\begin{figure}[ht]
%\centering
%\includegraphics[width=.5\textwidth]{fig1.jpg}
%\caption{A typical figure}
%\label{fig:exampleFig1}
%\end{figure}
%
%\begin{figure}[ht]
%\centering
%\includegraphics[width=.3\textwidth]{fig2.jpg}
%\caption{This figure is an example of a figure caption taking more than one
%  line and justified considering margins mentioned in Section~\ref{sec:figs}.}
%\label{fig:exampleFig2}
%\end{figure}


\bibliographystyle{sbc}
\bibliography{sbc-template}

\end{document}
